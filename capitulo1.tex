\chapter{Mecánica cuántica relativista}
\label{chap:MCR}

\section{Introducción}
\label{sec:IntroMCR}

La descripción de fenómenos físicos a altas energías requiere de una teoría cuántica consistente con la relatividad especial. La transición de una teoría no relativista a la relativista requiere de la re-formulación de algunos conceptos en la teoría no relativista, en particular:
\begin{enumerate}
	\item Las coordenadas temporales y espaciales deben ser tratadas de manera equivalente. Lo anterior no sucede en la en la mecánica cuántica no-relativista al aparecer $\frac{\partial}{\partial t}$ pero no $\overline{\nabla}$, sino $\nabla^2$ en la ecuación de Schrödinger.
	\item El principio de incertidumbre de Heisenberg implica que si la posición de una partícula es incierta entonces también el tiempo y la energía lo son:
	$$\Delta t\sim \frac{\Delta x}{c}\sim \frac{\hbar}{2c\Delta p }\sim \frac{\hbar}{2mc^2};$$ la posición de una partícula no puede ser determinada 
	con mayor precisión que su longitud de onda de Compton $\lambda_c=\frac{\hbar}{mc}$, de otro modo $E>2mc^2$ y se presenta la creación y aniquilación de pares.

	\item A energías relativistas se presenta la creación y aniquilación de partículas en pares de partícula y antipartícula, por lo que la conservación del número de partículas deja de ser válida. Lo anterior ocurre, por ejemplo, para partículas moviéndose a velocidades comparables a la velocidad de la luz.
\end{enumerate}

Como primer paso en la introducción de una teoría cuántica relativista seguimos el proceso histórico comenzando con el estudio de ecuaciones de onda relativistas de una partícula, i.e. ecuaciones de onda invariantes ante transformaciones de Lorentz\footnote{Éstas se introducen propiamente en la sección (\ref{sec:CovED})}.
%%%%%%%%%%%%%%%%%%%%%%%%%%%%%%%%%%%%%%%%%%%%%%%%%%%%%%%%%%%%%%%%%%%%%%%%%%%%%%%%%%%%%%%%%%%%%%%%%%%%%%%%%%%%%%
%%%%%%%%%%%%%%%%%%%%%%%%%%%%%%%%%%%%%%%%%%%%%%%%%%%%%%%%%%%%%%%%%%%%%%%%%%%%%%%%%%%%%%%%%%%%%%%%%%%%%%%%%%%%%%
\section{Ecuación de Klein-Gordon}
\label{sec:EcKleinGordon}

La energía clásica de una partícula no relativista con masa $m$, posición $\mathbf{x}$ y momento $\mathbf{p}$, sujeta a un potencial $V(\mathbf{x})$ es
\begin{equation}\label{EnergiaClasica}
\begin{array}{rl}
E&=H\\
&=\dfrac{\mathbf{p}^2}{2m}+V(\mathbf{x}).
\end{array}
\end{equation}
La ecuación (\ref{EnergiaClasica}) encapsula la dinámica entera del sistema; la utilización del formalismo apropiado ---i.e. Newton, Lagrange, Hamilton--- permite la determinación de $\mathbf{x}$ y $\mathbf{v}$, tras lo cual el sistema queda completamente descrito.\footnotemark
\footnotetext{En general $\mathbf{q}$ y $\dot{\mathbf{q}}$, ó $\mathbf{q}$ y $\mathbf{p}$.}

En la mecánica cuántica no-relativista un sistema ha sido enteramente descrito cuando se conoce $\Psi(\mathbf{x})$, solución de la \textit{ecuación de onda de Schrödinger}:
\begin{equation}\label{eq:Schrodinger}
i\hbar\frac{\partial}{\partial t}\Psi(\mathbf{x},t)=\left( -\frac{\hbar^2}{2m}\nabla^2 +V(\mathbf{x},t)\right)\Psi(\mathbf{x},t).
\end{equation}

La ecuación de Schrödinger (\ref{eq:Schrodinger}) puede también considerarse una consecuencia de (\ref{EnergiaClasica}) si en ésta se efectúa la siguiente transcripción operadores:
\begin{equation}\label{eq:TransOperadores}
\begin{array}{rl}
E\rightarrow \hat{E}&=i\hbar\dfrac{\partial}{\partial t}, \\[4pt] 
\mathbf{p}\rightarrow\hat{\mathbf{p}}&=-i\hbar\bar{\nabla}, \\[3.5pt]
H\rightarrow \hat{H}&=\dfrac{\hat{\mathbf{p}}^2}{2m}+V(\mathbf{x}).\\ 
\end{array}
\end{equation}

La mecánica descrita por (\ref{EnergiaClasica})---y en consecuencia por (\ref{eq:Schrodinger})--- es claramente no-relativista: basta observar que $E$ y $\mathbf{p}$ no aparecen en igualdad de condiciones; los distintos órdenes de las derivadas evidencain que esta expresión no puede ser invariante ante transformaciones de Lorentz. Lo anterior sugiere la posibilidad de obtener una ecuación de onda invariante ante transformaciones de Lorentz (de ahora en adelante, \textit{Lorentz-covariante}) partiendo de una relación relativista para la energía análoga a (\ref{EnergiaClasica}). De acuerdo con la relatividad especial, la energía de una partícula relativista libre de masa $m$ está dada por
\begin{equation}\label{EnergiaRel}
p^{\mu} p_{\mu} =\frac{E^2}{c^2}-\mathbf{p}\cdot\mathbf{p}=m^2c^2.
\end{equation}
Al sustituir $p^\mu$ por el operador de 4-momento $\hat{p}^\mu$,
\begin{equation}\label{Op4Momento}
\hat{p}^\mu =\left(\hat{p}_0,\hat{\mathbf{p}} \right)=
i\hbar\left( \dfrac{\partial}{\partial(ct)},-\dfrac{\partial}{\partial x},-\dfrac{\partial}{\partial y},-\dfrac{\partial}{\partial z} \right)
\end{equation}
se obtiene la ecuación de \textit{Klein-Gordon} para una partícula libre:
\begin{equation}\label{EcKleinGordon1}
\left( \hat{p}^\mu \hat{p}_\mu -m^2c^2\right)\psi=0.
\end{equation}
Es posible reescribir (\ref{EcKleinGordon1}) con la ayuda de (\ref{Op4Momento}) para obtener
\begin{equation}\label{EcKleinGordon2}
\left( \frac{\partial^2}{c^2 \partial t^2} -\frac{\partial^2}{\partial x^2}-\frac{\partial^2}{\partial y^2}-\frac{\partial^2}{\partial z^2}+\frac{m^2c^2}{\hbar^2} \right)\psi \equiv \left( \Box+\frac{m^2 c^2}{\hbar^2}\right) \psi=0.
\end{equation}
Se verifica de inmediato la Lorentz-covariancia de (\ref{EcKleinGordon1}) y (\ref{EcKleinGordon2}) al ser $\hat{p}^\mu\hat{p}_\mu$ Lorentz-covariante. 
Las soluciones libres de la ecuación de Klein-Gordon son ondas planas de la forma
\begin{equation}\label{SolLibKG}
\psi=N\exp{\left( -\frac{i}{\hbar}p_\mu x^\mu\right)}=N\exp{\left[ \frac{i}{\hbar}(\mathbf{p}\cdot \mathbf{x}-Et)\right]},
\end{equation}
con $N$ una constante de normalización. La inserción de (\ref{SolLibKG}) en la ecuación de Klein-Gordon arroja la relación
\begin{equation*}
p^{\mu}p_{\mu}=\frac{E^2}{c^2}-\mathbf{p}^2=m^2c^2, 
\end{equation*}
de donde
\begin{equation}\label{EnergiaKG}
 E=\pm \sqrt{m^2 c^4+\mathbf{p}^2c^2}.
\end{equation}

Lo anterior muestra que la ecuación de Klein-Gordon admite soluciones con energía tanto positiva como negativa. Las soluciones con energía negativa inicialmente representaron un obstáculo para la teoría, no obstante, éstas fueron posteriormente interpretadas con éxito en términos de antipartículas al reinterpretar (\ref{EcKleinGordon1}) en el marco de una teoría cuántica de campos.

Finalmente, se hace notar que es posible la construcción de una 4-corriente $j^\mu$:
\begin{equation}\label{CorrienteKG}
 j^\mu=(c\rho,-\mathbf{j})=\frac{i\hbar}{2m}(\psi^* \partial^\mu \psi-\psi \partial^\mu \psi^*),
\end{equation}
asociada a la ecuación de onda (\ref{EcKleinGordon1}), que de manera análoga al caso no relativista satisface
\begin{equation}\label{ContinuidadKG}
 \partial_\mu j^\mu=\frac{\partial \rho}{\partial t}+\nabla\cdot \mathbf{j}=0.
\end{equation}

%En (\ref{CorrienteKG}) y (\ref{ContinuidadKG}) $\partial_\mu\equiv\frac{\partial}{\partial x^\mu}$. 
La identidad (\ref{ContinuidadKG})  sugiere naturalmente 
la posibilidad de interpretar $\rho$ como una densidad de probabilidad; esta interpretación resulta imposible al observar que $\psi$ y $\frac{\partial \psi}{\partial t}$ 
pueden tomar valores arbitrarios, con lo cual $\rho$ no satisface ser positivo-definida. Esta situación, así como las antes mencionadas soluciones de energía negativa motivaron 
la búsqueda de ecuaciones de onda relativistas adicionales, algunas de las cuales se introducirán más adelante. Por último se menciona que la corriente $j_\mu$ se 
interpretó posteriormente y de manera exitosa como una corriente de carga $\pm ej_\mu$ asociada a tres soluciones correspondientes a partículas de carga positiva, 
negativa y neutra, para cada valor de $\mathbf{p}$.\cite[cap. 1]{GreinerRQM}

%%%%%%%%%%%%%%%%%%%%%%%%%%%%%%%%%%%%%%%%%%%%%%%%%%%%%%%%%%%%%%%%%%%%%%%%%%%%%%%%%%%%%%%%%%%%%%%%%%%%%%%%%%%%%%%%%%%%
%%%%%%%%%%%%%%%%%%%%%%%%%%%%%%%%%%%%%%%%%%%%%%%%%%%%%%%%%%%%%%%%%%%%%%%%%%%%%%%%%%%%%%%%%%%%%%%%%%%%%%%%%%%%%%%%%%%%

\section{Ecuación de Dirac}
\label{sec:EcDirac}

\subsection{Motivación}
\label{subsec:IntroduccionDirac}

En la sección (\ref{sec:EcKleinGordon}) se mencionaron algunas de las deficiencias que la ecuación de Klein-Gordon presenta como ecuación de onda relativista. En la ecuación de Schrödinger la derivada temporal $\hat{E}=i\hbar\partial_t$ aparece en un término lineal; la equivalencia relativista entre las coordenadas espaciales y temporales (sec. \ref{sec:IntroMCR}) requiere que toda ecuación de onda de la forma de Schrödinger $\hat{E}=\hat{H}$ Lorentz-covariante sea lineal en las derivadas espaciales. Con base en lo anterior en 1928 Paul M. Dirac propuso\cite{DiracQuantumTheoryElectron} una ecuación de la forma general

\begin{equation}\label{eq:EcDirac1}
i\hbar \frac{\partial \psi}{\partial t}= \hat{H}\psi =\frac{\hbar c}{i}\left( \alpha_1 \frac{\partial \psi}{\partial x^1} +\alpha_2\frac{\partial \psi}{\partial x^2} +\alpha_3 \frac{\partial \psi}{\partial x^3} \right)+ \beta m c^2 \psi.
\end{equation}

\noindent Es claro que en la ecuación (\ref{eq:EcDirac1}) los ---todavía no determinados--- coeficientes $\alpha_i$ no pueden ser escalares; de serlo ésta no sería invariante aún ante simples rotaciones espaciales. Dirac propuso (\ref{eq:EcDirac1}) como una ecuación matricial con $\alpha_i,\beta$ matrices de $N\times N$ y $\psi(\mathbf{x},t)$ un vector columna:
\begin{align}\label{eq:FuncOndDirac}
 \psi &=\left( \begin{matrix}
         \psi_1(\mathbf{x},t)\\
         \vdots\\
         \psi_N(\mathbf{x},t)\\
        \end{matrix}
        \right).
\end{align}
De esta manera, se convierte a la ecuación de Dirac en un sistema de $N$ ecuaciones diferenciales acopladas de primer orden. El valor de $N$ ---la dimensionalidad de $\psi$--- se determinará posteriormente. Las soluciones de (\ref{eq:EcDirac1}) de la forma (\ref{eq:FuncOndDirac}) se denominan \textit{espinores} en analogía con las soluciones de la ecuación de Pauli para partículas de espín $\sfrac{1}{2}$.\footnotemark Para considerar a (\ref{eq:EcDirac1}) como una ecuación de onda físicamente aceptable requerimos que se satisfaga lo siguiente:

\begin{enumerate}[\hspace{15pt}(a)]
 \item La relación energética relativista para partículas libres
 \begin{equation*}
  E^2=\mathbf{p}^2m^2+m^2c^4.
 \end{equation*}

 \item Admisión de una 4-corriente $j^\mu$ con su correspondiente ecuación de continuidad $\partial_\mu j^\mu=0$. El término temporal de $j^\mu$ debe ser positivo-definido.
 
 \item Covariancia de Lorentz, i.e. la forma de (\ref{eq:EcDirac1}) debe ser invariante frente a transformaciones de Lorentz de un sistema inercial a otro.
 \end{enumerate}

\footnotetext{La ecuación de Dirac se reduce en el límite no relativista a la ecuación de Pauli. Si bien esto se argumentará de manera distinta, este hecho permite concluir que la ecuación de Dirac describe a partículas de espín $\sfrac{1}{2}$; ver \cite[p.~10]{Bjorken}.}

La condición (a) equivale a requerir que cada componente de (\ref{eq:FuncOndDirac}) satisfaga la ecuación de Klein-Gordon (ver sec. \ref{sec:EcKleinGordon}):

\begin{equation}\label{eq:KGparaDirac}
 -\hbar^2\frac{\partial^2 \psi_\sigma}{\partial t^2}=(-\hbar^2c^2\nabla^2+m^2c^4)\psi_\sigma.
\end{equation}

La composición del hamiltoniano $H$ de la ecuación de Dirac arroja:
\begin{equation}\label{eq:DobleDirac}
\begin{aligned}
 -\hbar^2 \frac{\partial^2 \psi_\sigma}{\partial t^2} &= \mathlarger{\mathlarger{\sum}}_{i,j=1}^3\frac{\alpha_i \alpha_j+\alpha_j \alpha_i}{2}\frac{\partial^2 \psi_\sigma}{\partial x^i \partial x^j} \\
   & \quad -i\hbar mc^3 \mathlarger{\mathlarger{\sum}}_{i=1}^3 (\alpha_i \beta+\beta \alpha_i)\frac{\partial \psi_\sigma}{\partial x^i}+\beta^2 m^2 c^2\psi_\sigma.
\end{aligned}
\end{equation}
Las expresiones (\ref{eq:KGparaDirac}) y (\ref{eq:DobleDirac}) sólo coincidirán si se exije adicionalmente que las matrices $\alpha_i$, $\beta$ satisfagan
\begin{subequations}\label{eq:RelAnticonAlfa}
\begin{align}
 \left\{ \alpha_i,\alpha_j \right\} \equiv \alpha_i\alpha_j &+\alpha_j\alpha_i =0\quad (i\neq j),\label{eq:RelAnticonAlfaA}\\[8pt] 
 \left\{ \alpha_i,\beta \right\} \equiv \alpha_i\beta &+\beta\alpha_i =0, 
 \label{eq:RelAnticonAlfaB}\\[8pt]
 {\alpha_i}^2=& \beta^2=\mathbb{1} \quad (\text{Matriz identidad } N\times N)\label{eq:RelAnticonAlfaC}
\end{align}
\end{subequations}
Las relaciones en (\ref{eq:RelAnticonAlfa}) se denominan relaciones de anticonmutación. El requerimiento adicional de que $\alpha_i$ y $\beta$ sean hermitianas ---siendo así $H$ hermitiano--- determina $\alpha_i$ y $\beta$ hasta una transformación unitaria; los valores propios de $\alpha_i$ y $\beta$ deben ser reales, y por (\ref{eq:RelAnticonAlfaC}) éstos deben ser $\pm1$. La propiedad (\ref{eq:RelAnticonAlfaB}) implica que 
\begin{equation*}
     -\alpha_i=\beta \alpha \beta
\end{equation*}
y por lo tanto
\begin{equation}\label{eq:TrazaAlfaBeta}
 \Tr(\alpha_i)=\Tr(\beta\beta \alpha_i)=\Tr(\beta \alpha_i \beta)=\Tr(-\alpha_i),
\end{equation}
de donde se obtiene
\begin{equation}\label{eq:TrazaCero}
     \Tr(\alpha_i)=\Tr(\beta)=0.
\end{equation}

En (\ref{eq:TrazaAlfaBeta}) se utilizó $\Tr(AB)=\Tr(BA)$. La ecuación (\ref{eq:TrazaCero}) nos dice que las matrices $\alpha_i$ y $\beta$ deben tener el mismo número de valores propios positivos ($+1$) y negativos ($-1$) en su diagonal, por lo que $N$ ---la dimensión de $\psi$--- debe ser par. En el caso $N=2$ el máximo número de matrices que satisfacen (\ref{eq:RelAnticonAlfaA}-\ref{eq:RelAnticonAlfaC}) es 3, e.g. las matrices de Pauli $\sigma_i$.\footnotemark El mínimo $N$ para el cual (\ref{eq:RelAnticonAlfa}) se satisface es $N=4$. Una representación explícita de las matrices $\alpha_i, \beta$ es
\begin{equation}\label{eq:MatricesAlfaBeta}
     \alpha_i=\begin{pmatrix}
		0 & \sigma_i \\
		  \sigma_i & 0
	      \end{pmatrix},\quad
     \beta=\begin{pmatrix}
                \mathbb{1} & 0 \\
                0 & -\mathbb{1}
           \end{pmatrix}.
\end{equation}

\footnotetext{Un conjunto de matrices $A_i$ que satisfagan (\ref{eq:RelAnticonAlfa}) debe ser linealmente independiente y satisfacer $\Tr(A_i)=0$ para toda $i$. La dimensión del subespacio de matrices $M$ de $2\times 2$ para las cuales $\Tr(M)=0$ es 3, por lo que esa es la máxima cantidad de matrices $2\times2$ que satisfacen (\ref{eq:RelAnticonAlfa}).} 

La representación (\ref{eq:MatricesAlfaBeta}) es sólo una de las múltiples representaciones admisibles; cualquier conjunto ${\alpha'}_i=U\alpha_i U^{-1}, \beta'=U\beta U^{-1}$ con $U$ una matriz unitaria así mismo satisfará (\ref{eq:RelAnticonAlfa}). La siguiente subsección provee mayores detalles sobre la relación entre distintas representaciones.

\subsection{La ecuación de Dirac y las matrices \texorpdfstring{$\gamma^\mu$}{gamma}}
\label{subsec:MatricesGamma}

La invarianza relativista de la ecuación de Dirac (sec. \ref{sec:CovED}) no es evidente en la forma de Schrödinger $i\hbar \frac{\partial \psi}{\partial t}= \hat{H}\psi$ (\ref{eq:EcDirac1}); se observa una clara distinción entre el término temporal y los términos espaciales. En la subsecuente discusión resultará beneficioso re-expresar (\ref{eq:EcDirac1}) de manera explícitamente covariante, por lo que multiplicamos (\ref{eq:EcDirac1}) por $\frac{\beta}{c}$ y definimos 
\begin{equation}\label{eq:MatGammaDef}
	\gamma^0=\beta,\quad \gamma^i=\beta\alpha_i.
\end{equation}
La ecuación de Dirac resulta
\begin{equation}\label{eq:EDbis}
	i\hbar\left(\gamma^0 \frac{\partial}{\partial x^0}+\gamma^1 \frac{\partial}			{\partial x^1}+\gamma^0 \frac{\partial}{\partial x^2}+\gamma^2 						\frac{\partial}{\partial x^3} \right)\psi + mc\psi=0.
\end{equation}
Introduciendo la notación de barra (\textit{slash}) de Feynman para 4-vectores: $\slashed{a}=a_{\mu}\gamma^{\mu}$, así como unidades naturales $\hbar=c=1$, la ecuación de Dirac se escribe de manera compacta como:
\begin{equation}\label{eq:ED}
	(i\slashed{\partial}-m)\psi=(\slashed{p}-m)\psi=0.
\end{equation}
En la representación (\ref{eq:MatricesAlfaBeta}) las matrices $\gamma^{\mu}$ son:
\begin{equation}\label{eq:MatGammaRep}
     \gamma^0=\begin{pmatrix}
                \mathbb{1} & 0 \\
                0 & -\mathbb{1}
           \end{pmatrix},\quad
     \gamma^i=\begin{pmatrix}
		0 & \sigma_i \\
		  -\sigma_i & 0
	      \end{pmatrix},
\end{equation}
y satisfacen las relaciones de anticonmutación y hermiticidad\footnotemark
\begin{subequations}\label{eq:AlgClifford}
\begin{align}
 \left\{ \gamma^{\mu},\gamma^{\nu}\right\}=\gamma^{\mu}\gamma^{\nu} &+\gamma^{\nu}\gamma^{\mu} =g^{\mu \nu}\mathbb{1},\label{eq:AnticonmutacionClifford}\\[8pt] 
 (\gamma^{0})^{\dagger}=\gamma^{0},\; & \; (\gamma^{i})^{\dagger}=-\gamma^{i} \label{HermClifford}\\[8pt]
 (\gamma^{0})^2 = \mathbb{1},\,&\,(\gamma^{i})^2 = -\mathbb{1}.\label{eq:UnitClifford}
\end{align}
\end{subequations}

\footnotetext{Se dice que las matrices $\gamma^\mu$ forman un \textit{álgebra de Clifford}.\cite{ZeeQTFNut} Sus conmutadores son generadores del grupo de Lorentz,\citep{RobinsonSymmetry} este hecho se usará para mostrar la Lorentz-covariancia de la ecuación de Dirac en la sección \ref{sec:CovED}.}

El objeto $g$ que aparece en (\ref{eq:AlgClifford}) es la generalización relativista de la métrica euclidiana $\delta_{i j}$ y se conoce como \textit{métrica de Minkowski} o simplemente \textit{tensor métrico} [relativista]. Matemáticamente:
\begin{equation}\label{eq:Metrica} 
g^{\mu\nu}=	\begin{pmatrix}
			1 & 0 & 0 & 0 \\
			0 & -1 & 0 & 0 \\
			0 & 0 & -1 & 0 \\
			0 & 0 & 0 & -1 \\
			\end{pmatrix}
\end{equation}
La métrica $g$ encapsula la <<estructura>> relativista del espacio-tiempo y permite la construcción de objetos físicos invariantes en todo sistema de referencia inercial (sec. \ref{sec:CovED}).

El mismo argumento utilizado para $\alpha_i$ y $\beta$ muestra que cualquier conjunto de matrices ${\gamma'}^{\mu}$ unitariamente equivalentes ---i.e. ${\gamma'}^{\mu}= U\gamma^{\mu}U^{-1}$--- a (\ref{eq:MatGammaRep}) satisfará así mismo (\ref{eq:AlgClifford}). Inversamente, en las referencias \cite{Good,GreinerRQM} se muestra que todas las familias de matrices $4\times 4$ que satisfacen (\ref{eq:AlgClifford}) son unitariamente equivalentes. De lo anterior se desprende el que las conclusiones físicas obtenidas de la ecuaciónes (\ref{eq:EcDirac1}) y (\ref{eq:ED}) serán independientes de la selección particular de $\alpha_i$ y $\beta$, y $\gamma^{\mu}$, respectivamente, mas la selección juiciosa de dicha representación posibilita la simplificación de los cálculos involucrados.

La forma (\ref{eq:ED}) de la ecuación de Dirac permite la inmediata inserción de una interacción con un potencial externo $A^{\mu}$ mediante la <<sustitución mínima>> $p^{\mu}\rightarrow p^{\mu}-eA^{\mu}$:\footnotemark
\begin{equation}\label{eq:EDI}
	(\gamma_{\mu}(p^{\mu}-eA^{\mu})-m)\psi=(i\slashed{\partial}-e\slashed{A}-m)\psi=0.
\end{equation}
\footnotetext{También llamado <<acoplamiento mínimo>>.}
En el contexto de la invarianza relativista de la ecuación de Dirac, si $p^{\mu}$ es un 4-vector entonces $p^{\mu}-eA^{\mu}$ también lo será, por lo que en (\ref{sec:CovED}) bastará verificar la covariancia de Lorentz de (\ref{eq:ED}); la covariancia de (\ref{eq:EDI}) será una consecuencia directa del primer caso. 

\subsection{Ecuación de continuidad}
\label{subsec:ContinuidadDirac}

El carácter vectorial de las soluciones de (\ref{eq:EcDirac1}), (\ref{eq:FuncOndDirac}) sugiere la utilización de formas bilineares en $\psi$ para 
la construcción de una corriente conservada $(\rho,\mathbf{j})$ con componente temporal $\rho$ positiva-definida.\footnote{La palabra \textit{vector} se utiliza aquí en el sentido usual de álgebra lineal; físicamente $\psi$ es un \textit{espinor}.} En efecto, partiendo de la ecuación de Dirac (\ref{eq:EcDirac1}) y su transpuesta conjugada:\footnotemark
\footnotetext{Para una matriz arbitraria $M_{n\times m}$: $M^\dagger={(M^T)}^\ast$. En particular: $\psi^\dagger=\left({\psi_1}^\ast,\ldots,{\psi_4}^\ast \right)$.}
\begin{subequations}\label{EcDyEcDD}
    \begin{equation}\label{EcD}
      i\hbar \frac{\partial \psi}{\partial t} =\frac{\hbar c}{i}\left( \alpha_1 \frac{\partial \psi}{\partial x^1} +\alpha_2\frac{\partial \psi}{\partial x^2} +\alpha_3 \frac{\partial \psi}{\partial x^3} \right)+ \beta m c^2 \psi,\\[8pt]
    \end{equation}
    \begin{equation}\label{EcDD}
      -i\hbar \frac{\partial \psi^\dagger}{\partial t} =-\frac{\hbar c}{i}\left( {\alpha_1}^\dagger \frac{\partial \psi^\dagger}{\partial x^1} +{\alpha_2}^\dagger\frac{\partial \psi^\dagger}{\partial x^2} +{\alpha_3}^\dagger \frac{\partial \psi^\dagger}{\partial x^3} \right)+ \beta^\dagger m c^2 \psi^\dagger.
    \end{equation}
\end{subequations}
La multiplicación de (\ref{EcD}) por $\psi^\dagger$ en la derecha, de (\ref{EcDD}) por $\psi$ en la izquierda y la sustracción de lo resultante arroja
\begin{equation}\label{ContinuidadCompletaAlfa}
     i\hbar \frac{\partial}{\partial t}\psi^\dagger \psi=\sum_1^3\frac{\hbar c}{i}\frac{\partial}{\partial x_i}\left(\psi^\dagger \alpha_i \psi\right),
\end{equation}
o equivalentemente
\begin{equation}\label{ContinuidadAlfa}
     \frac{\partial}{\partial t}\rho+\nabla\cdot \mathbf{j}=0,
\end{equation}
con densidad $\rho=\psi^\dagger \psi=\sum_1^4 \psi_i^{\ast}\psi_i\geq 0$ definida-positiva y $\mathbf{j}=c\psi^\dagger \mathbf{\alpha}_i \psi$. 

La expresión (\ref{ContinuidadAlfa}) puede reescribirse en términos de las matrices $\gamma^{\mu}$ (ver sub. \ref{subsec:MatricesGamma}) introduciendo $\bar{\psi}=\psi^{\dagger}\gamma^0$ el \textit{espinor adjunto} de $\psi$. En unidades naturales ($\hbar=c=1$):
\begin{equation}
	j^{\mu}=\bar{\psi}\gamma^{\mu}\psi,
\end{equation}
de manera que (\ref{ContinuidadAlfa}) se convierte simplemente en 
\begin{equation}\label{eq:ContinuidadGamma}
	\partial_{\mu}j^{\mu}=0.
\end{equation}
En la sección \ref{sec:CovED} se mostrará que $j^{\mu}$ es un 4-vector, haciendo así a la ecuación de continuidad (\ref{eq:ContinuidadGamma}) un invariante de Lorentz.

%%%%%%%%%%%%%%%%%%%%%%%%%%%%%%%%%%%%%%%%%%%%%%%%%%%%%%%%%%%%%%%%%%%%%%%%%%%%%%%%%%%%%%%%%%%%%%%%%%%%%%%%%%%%%%%%%%%
%%%%%%%%%%%%%%%%%%%%%%%%%%%%%%%%%%%%%%%%%%%%%%%%%%%%%%%%%%%%%%%%%%%%%%%%%%%%%%%%%%%%%%%%%%%%%%%%%%%%%%%%%%%%%%%%%%%

\section{Covariancia de la ecuación de Dirac}
\label{sec:CovED}

\subsection{Transformaciones de Lorentz}
\label{subsec:TransfLorentz}

La relatividad especial establece la existencia de una clase de sistemas de referencia, llamados \textit{inerciales}, completamente equivalentes para la descripción de fenómenos físicos; dos observadores en sistemas de referencia inerciales $O$ y $O'$ con coordenadas $x^{\mu}=(t,x,y,z)$ y ${x'}^{\mu}=(t',x',y',z')$, no necesariamente iguales, llegarán a las mismas conclusiones tras la observación de un fenómeno físico.\footnotemark
\footnotetext{$x^{\mu}$ es un \textit{evento} en el \textit{espacio-tiempo} o \textit{espacio de Minkowski} $3+1$-dimensional. En lo subsecuente se adopta la convención $c=\hbar=1$, anteriormente utilizada en la subsección \ref{subsec:MatricesGamma}.}
La relación entre las coordenadas $x^{\mu}$ y ${x'}^{\mu}$ de ambos sistemas de referencia está dada por
\begin{equation}\label{eq:TransfLorentz}
	{x^\prime}^{\mu}=\sum_{\nu=0}^4 \Lambda^{\mu}{}_{\nu} x^{\nu}\equiv \Lambda^{\mu}{}_{\nu}x^{\nu},\footnotemark
\end{equation}
sujeta a la restricción 
\begin{equation}\label{eq:CondLorentz}
\Lambda^\mu{}_\sigma \Lambda^\nu{}_\tau g^{\sigma \tau}=g^{\mu \nu}.
\end{equation}

El objeto $g$ que aparece en (\ref{eq:CondLorentz}) es el \textit{tensor métrico} [relativista] o la \textit{métrica de Minkwoski} ---ya previamente encontrado en la sección \ref{subsec:IntroduccionDirac}--- y se define por:
\begin{equation}\label{eq:Metrica} 
g^{\mu\nu}=	\begin{pmatrix}
			1 & 0 & 0 & 0 \\
			0 & -1 & 0 & 0 \\
			0 & 0 & -1 & 0 \\
			0 & 0 & 0 & -1 \\
			\end{pmatrix}.
\end{equation}
La condición (\ref{eq:CondLorentz}) garantiza la constancia del <<producto interior>>: $u_{\mu}v^{\mu}=g_{\mu\nu}u^{\nu}v^{mu}$, y la <<norma>> o \textit{intervalo invariante} inducida por éste: $u^2=u_\mu u^\mu$, en ambos sistemas de referencia; la constancia de la velocidad de la luz se sigue directamente del caso particular en el que $x^2=0={x^\prime}^2$. La transformación lineal y homogénea $\Lambda$ en (\ref{eq:TransfLorentz}) y (\ref{eq:CondLorentz}) lleva por nombre el de  \textit{transformación de Lorentz}. Los coeficientes $\Lambda^{\mu}{}_{\nu}$ de $\Lambda$ dependen de las velocidades y orientaciones relativas de los sistemas coordenados de $O$ y $O'$. 
\footnotetext{Como se describe en la sección sobre notación, a reserva de especificar lo contrario, se hace uso de la convención de suma de Einstein en la que índices repetidos se suman.}

Es físicamente claro que para todo sistema de referencia $O$ existe una transformación $\Lambda$ ---en este caso la identidad $\mathbb{1}$--- que conecta el sistema coordenado de $O$ consigo mismo, y que si $\Lambda_1$ conecta a $O$ con $O^\prime$ y $\Lambda_2$ conecta $O^\prime$ con $O^{\prime\prime}$, entonces $\Lambda_1 \Lambda_2$ realiza el cambio de coordenadas de $O$ a $O^{\prime \prime}$. La simetría del sistema ---con respecto a los observadores---sugiere además que si $\Lambda$ conecta la descripción de $O$ con la de $O^\prime$, entonces debe existir una transformación inversa $\Lambda^{-1}$ que realice la conexión en el sentido inverso. De lo anterior se extrae que el conjunto de todas las transformaciones de Lorentz forma un grupo, el \textit{grupo de Lorentz}, denotado $\text{SO}(1,3)$.\footnote{La notación $\text{SO}(1,3)$ proviene del hecho de que el grupo es <<parecido>> a $\text{SO}(4)$: en $\text{SO}(4)$ la métrica es $\delta_{\mu \nu}$, que induce la norma euclidiana usual en cuatro dimensiones; en el grupo de Lorentz la métrica $g$ difiere de $\delta$ en el cambio de signo entre los términos espaciales y temporales del intervalo invariante $d^2 x $; $\text{SO}(4)$ y $\text{SO}(1,3)$ exhiben la misma <<geometría>> tras la sustitución de las rotaciones circulares del tiempo y el espacio por <<rotaciones>> hiperbólicas. \citep{RobinsonSymmetry}} El grupo de Lorentz contiene como subgrupo a las isometrías usuales del espacio tridimensional ---los grupos de rotaciones $\text{SO}(2), \text{SO}(3)$ y las reflexiones o inversiones del espacio---, las inversiones temporales $(t\mapsto -t)$ y los \textit{boosts} ---si bien éstos no forman un grupo--- que transforman un sistema inercial en otro con una velocidad relativa con respecto a éste.\footnotemark

\footnotetext{El grupo de Lorentz es, a su vez, subgrupo del \textit{grupo de Poincaré} o \textit{grupo inhomogéneo de Lorentz}, el cual contiene además transformaciones de la forma ${x^\prime}^{\mu}=\Lambda^{\mu}{}_{\nu}x^{\nu}+a^\mu $.
Es claro que la transformación de la ecuación de Dirac y sus soluciones $\psi$ ante una traslación  $x^\mu+a^\mu$ es un simple cambio de coordenadas, por lo que las propiedades de transformación estudiadas en la presente sección se restringirán a transformaciones homogéneas de Lorentz.}

\subsection{Propiedades de transformación de la ecuación de Dirac}
\label{subsec:PropTransEcDirac}

Dados dos observadores inerciales $O$ y $O'$ con sus respectivos sistemas coordenados $x^\mu$ y ${x^\prime}^\mu$, conectados por una transformación de Lorentz $(x^\prime=\Lambda x)$, el requerimiento de que la ecuación de Dirac sea Lorentz-covariante se traduce en las siguientes condiciones:
\begin{enumerate}[(a)]
	\item Si $\psi(x)$, solución de (\ref{eq:EDbis}), describe un estado físico en el sistema $O$ y $\psi^{\prime}(x^{\prime})$ describe el mismo estado físico en el sistema $O'$, debe existir una transformación $S(\Lambda)$ que permita el cálculo de $\psi^{\prime}$ a partir de $\psi$, y viceversa.
%	\item La equivalencia entre los sistemas $O$ y $O'$ implica que 
\item $\psi^{\prime}$ debe satisfacer la correspondiente ecuación de Dirac en el sistema $O'$:
	\begin{equation}\label{eq:DiracPrima}
	i\left({\gamma^{\prime}}^0 \frac{\partial}{\partial {x^{\prime}}^0}+{\gamma^{\prime}}^1 \frac{\partial}{\partial {x^{\prime}}^1}+{\gamma^{\prime}}^2 \frac{\partial}{\partial {x^{\prime}}^2}+{\gamma^{\prime}}^3 						\frac{\partial}{\partial {x^{\prime}}^3} \right)\psi^{\prime}(x^{\prime}) + m\psi^{\prime}(x^{\prime})=0.
\end{equation}
\end{enumerate}
Adicionalmente, el conjunto de matrices $\gamma^{\prime}$ en $O^\prime$ debe satisfacer el álgebra de Clifford (\ref{eq:AlgClifford}), de lo contrario sería posible distinguir los sistemas $O$ y $O'$. La referencia \citep{Good} mencionada en la subsec. \ref{subsec:MatricesGamma} establece que todas las matrices $4\times 4$ que satisfacen (\ref{eq:AlgClifford}) son unitariamente equivalentes; se desprende que es posible prescindir de las primas en (\ref{eq:DiracPrima}) y utilizar el mismo conjunto de matrices de Dirac $\gamma$.

La linealidad de las transformaciones de Lorentz y de la ecuación de Dirac sugiere que la transformación $S(\Lambda)$ debe ser lineal; se busca entonces una matriz $4\times 4$ $S(\Lambda)$ tal que
\begin{equation}\label{eq:TransfS}
\psi^{\prime}(x^{\prime})=\psi^{\prime}(\Lambda x)=S(\Lambda)\psi(x)=S(\Lambda)\psi(\Lambda^{-1}x^{\prime}).
\end{equation}

La equivalencia entre los dos observadores inerciales requiere que $O'$ sea capaz de calcular $\psi(x)$ a partir de $\psi^{\prime}(x^{\prime})$; la transformación $S(\Lambda)$ debe tener inversa $S^{-1}(\Lambda)$. Por otro lado, $\Lambda^{-1}$  lleva a cabo la transformación $O'\rightarrow O$.  Lo anterior, en conjunto con (\ref{eq:TransfS}), permite realizar la identificación
\begin{equation}\label{eq:TransfInv}
S(\Lambda^{-1})=S^{-1}(\Lambda).\footnotemark
\end{equation}
\footnotetext{Esto es parte del más general hecho de que $S$ forma una \textit{representación} (espinorial) del grupo (homogéneo) de Lorentz, esto se explora con detenimiento en el capítulo 2 de \cite{WeinbergQTF}.}
La condición que determina $S$ se deduce de aplicar $S(\Lambda)$ por la izquierda en (\ref{eq:EDbis}) y escribir $\psi(x)$ como $S(\Lambda^{-1})\psi^{\prime}(x^{\prime})$:
\begin{equation}\label{eq:EDintermedia1}
	\left( iS(\Lambda)\gamma^{\mu}S^{-1}(\Lambda)\frac{\partial}{\partial x^{\mu}}-m\right)\psi^{\prime}(x^{\prime})=0.
\end{equation} 
Adicionalmente, de (\ref{eq:TransfLorentz}) se deduce que $\dfrac{\partial}{\partial x^\mu}=\dfrac{\partial {x^{\prime}}^{\nu}}{\partial x^{\mu}}\dfrac{\partial}{\partial {x^{\prime}}^{\nu}}=\Lambda^{\nu}{}_{\mu}\dfrac{\partial}{\partial {x^{\prime}}^{\nu}}$; se re-escribe entonces (\ref{eq:EDintermedia1}) como:
\begin{equation}\label{eq:EDintermedia2}
	\left( iS(\Lambda)\gamma^{\mu}S^{-1}(\Lambda)\Lambda^{\nu}{}_{\mu}\frac{\partial}{\partial x^{\nu}}-m\right)\psi^{\prime}(x^{\prime})=0.
\end{equation}
La identificación de (\ref{eq:EDintermedia2}) con la ecuación de Dirac para $\psi^{\prime}(x^{\prime})$, (\ref{eq:EDbis}) permite concluir la condición fundamental sobre $S$:
\begin{equation}\label{eq:CondS}
\Lambda^{\nu}{}_{\mu}\gamma^{\mu}=S^{-1}(\Lambda)\gamma^{\nu}S(\Lambda).
\end{equation}

El la existencia de una solución $S$ de (\ref{eq:CondS}) constituirá la prueba de la Lorentz-covariancia de la ecuación de Dirac; las condiciones (a) y (b) para ésta se satisfarán de manera inmediata por una $S$ con dicha propiedad. 

El problema de encontrar una solución $S$ de (\ref{eq:CondS}) se simplifica enormemente para transformaciones de Lorentz <<pequeñas>> o cercanas a la identidad $a^{\mu}{}_{\nu}$:\footnotemark
\begin{equation}\label{eq:LorentzInf}
a^{\mu}{}_{\nu}=\delta^{\mu}_{\nu}+\Delta \omega^{\mu}{}_{\nu},
\end{equation}
con $\Delta \omega$ una transformación antisimétrica: $\Delta \omega^{\mu \nu}=-\Delta \omega^{\nu \mu}$. La antisimetría de $\Delta \omega^{\mu \nu}$ es consecuencia de la invertibilidad de la transformación $a$:
\begin{align*}
a^{\mu}{}_{\nu}a_{\mu}{}^{\lambda}= \delta_{\nu}^{\lambda}&= (\delta^{\mu}_{\nu}+\Delta \omega^{\mu}{}_{\nu})(\delta^{\lambda}_{\mu}+\Delta \omega_{\mu}{}^{\lambda}) \\
& \approx  \delta_{\nu}^{\lambda}+\Delta \omega^{\lambda}{}_{\nu} +\Delta \omega_{\nu}{}^{\lambda}.
\end{align*}
al ignorar términos de segundo orden. La antisimetría de la matriz $\Delta \omega^{\mu \nu}$ restringe la cantidad de sus coeficientes independientes a seis: las tres entradas $(\Delta \omega)^{0i}=\Delta v_i$ generarán los \textit{boosts} con velocidad $\Delta v_i$ en la dirección $x_i$; las tres entradas restantes $(\Delta \omega)^{ij}=\Delta \varphi_k$ generarán rotaciones de ángulo $\Delta \varphi_k$ alrededor del eje $x_k$ ---$i$, $j$ y $k$ cíclicos---. $\mathbf{v}$ y $\bm{\varphi}$ se denominan \textit{ángulos de rotación generalizados} $\omega$ y comprenden los ángulos de rotación usuales en tres dimensiones, así como los <<ángulos>> de los \textit{boosts} ---rotaciones hiperbólicas--- que mezclan coordenadas espaciales y temporales.
\footnotetext{Esto nos obliga a trabajar en la región conexa del grupo de Lorentz que contiene a la transformación identidad: el grupo de las transformaciones de Lorentz \textit{propias}. El grupo de las transformaciones propias satisface $\det{\Lambda}=+1$, para toda $\Lambda$. Las transformaciones $\Lambda$ para las cuales $\det{\lambda}=-1$ se denominan transformaciones de Lorentz \textit{impropias}; las inversiones espaciales y las temporales son ejemplos de éstas. }

Dada una transformación de Lorentz $a$ cercana a la identidad, resulta físicamente razonable asumir que la transformación $S(a)$ lo estará también; en otras palabras, que la dependencia de $S$ en $a$ es continua. Matemáticamente lo anterior equivale  a la asunción de que $S(a)$ admite una expansión \textit{a primer orden} en potencias de $\Delta \omega$:
\begin{equation}\label{eq:ExpansionSinfinitesimal}
S(a)=\mathbb{1}-\frac{i}{4}\sigma_{\mu \nu}\Delta \omega^{\mu \nu}
\end{equation}
con $$\sigma_{\mu \nu}=-\sigma_{\nu \mu}.$$
Cada uno de los seis <<coeficientes>> $\sigma_{\mu \nu}$ en (\ref{eq:ExpansionSinfinitesimal}) es una matriz $4\times 4$ y éstos se conocen como los \textit{generadores} de la representación $S$. El factor $-\frac{i}{4}$ se incluye para simplificar cálculos posteriores. Insertando (\ref{eq:LorentzInf}) y (\ref{eq:ExpansionSinfinitesimal}) en la condición fundamental (\ref{eq:CondS}) se obtiene, a primer orden en $\Delta \omega$:
\begin{equation*}
\Delta \omega^{\nu}{}_{\mu}\gamma^{\mu}=-\frac{i}{4}\Delta \omega^{\alpha \beta}(\gamma^\nu \sigma_{\alpha \beta}-\sigma_{\alpha \beta}\gamma^{\nu}).
\end{equation*}
La antisimetría de $\Delta \omega$ permite la manipulación del lado izquierdo de lo anterior para obtener la condición que las matrices $\sigma_{\alpha \beta}$ han de satisfacer:
\begin{equation}\label{eq:CondCoefSigma}
\left[ \gamma^\mu ,\sigma_{\alpha\beta}\right]=2i(g^{\mu}{}_{\alpha}\gamma_{\beta}-g^{\mu}{}_{\beta}\gamma_{\alpha}).
\end{equation}
Para transformaciones de Lorentz cercanas a la identidad, el problema de hallar $S$ que satisfaga (\ref{eq:CondS}) queda entonces reducido al de encontrar un conjunto de seis matrices que satisfagan (\ref{eq:CondCoefSigma}). De la relación de anticonmutación (\ref{eq:AnticonmutacionClifford}) se sigue que 
\begin{equation}\label{eq:DefSigma}
\sigma_{\mu \nu}\equiv \frac{i}{2}\left[\gamma_\mu,\gamma_\nu \right]
\end{equation}
satisface (\ref{eq:CondCoefSigma}) y por tanto es el conjunto de matrices buscado. La transformación $S(a)$ con $a$ \textit{infinitesimal} es entonces:
\begin{equation}\label{eq:Sinfinitesimal}
S(a)=\mathbb{1}-\frac{i}{2}\sigma_{\mu \nu}\Delta \omega^{\mu \nu}=\mathbb{1}+\frac{1}{8}\left[\gamma_\mu,\gamma_\nu \right]\Delta \omega^{\mu \nu}
\end{equation}

La construcción de $S(\Lambda)$ para transformaciones de Lorentz $\Lambda$ con un ángulo de rotación finito $\omega$ se efectúa mediante la sucesiva aplicación de la transformación infinitesimal (\ref{eq:Sinfinitesimal}). Para realizar lo anterior resulta útil re-escribir (\ref{eq:LorentzInf}) como
\begin{equation}\label{LorentzInfBis}
a^{\mu}{}_{\nu}=\delta^{\mu}_{\nu}+\Delta \omega^{\mu}{}_{\nu}=\delta^{\mu}_{\nu}+\Delta \omega (I_n)^{\mu}{}_{\nu}
\end{equation}
donde se expresa a $a$ como una rotación infinitesimal de ángulo generalizado $\Delta \omega$ en la dirección $\bm{n}$ en el espacio de Minkowski; $(I_n)^{\mu}{}_{\nu}$ es la matriz $4\times 4$ que genera dicha rotación unitaria en la dirección $\bm{n}$. Las transformaciones finitas correspondientes se obtienen a través una sucesión de aplicaciones de (\ref{LorentzInfBis}) mediante
\begin{equation}\label{LorentzExp}
\Lambda = \lim_{N\rightarrow \infty}\left( \mathbb{1}+\frac{\omega}{N}I_n\right)^N= \exp{\left(\omega I_n\right)} \qquad \left(\Delta \omega = \frac{\omega}{N}\right)
\end{equation}
La transformación para espinores $S(\Lambda)$ con $\Lambda$ finita se sigue de (\ref{eq:TransfS}), (\ref{LorentzInfBis}) y (\ref{LorentzExp}):
\begin{equation}\label{eq:SFinita}
\begin{aligned}
\psi^{\prime}(x^{\prime})=S(\Lambda)\psi(x)&=\lim_{N\rightarrow \infty}\left( \mathbb{1}-\frac{i}{4}\frac{\omega}{N}\sigma_{\mu \nu}I_{n}^{\mu \nu}\right)^N \psi(x)\\
&=\,\exp{\left( -\frac{i}{4}\omega\sigma_{\mu \nu}I_{n}^{\mu \nu} \right)} \psi(x)
\end{aligned}
\end{equation}
Resulta ilustrativo ejemplificar lo anterior:
\begin{example} \textit{Boost} de velocidad $v$ en la dirección $x$. En este caso la transformación infinitesimal es
	\begin{equation}\label{BoostInfX}
		a^{\mu}{}_{\nu}=	\begin{pmatrix}
							1 & -\Delta v & 0 & 0 \\
							-\Delta v & 1 & 0 & 0 \\
							0 & 0 & 1 & 0 \\
							0 & 0 & 0 & 1	
							\end{pmatrix}
		= \mathbb{1}+\Delta v 	\begin{pmatrix}
								0 & -1 & 0 & 0\\
								-1 & 0 & 0 & 0 \\
								0 & 0 & 0 & 0 \\
								0 & 0 & 0 & 0
								\end{pmatrix}
	\end{equation}
	de donde se deduce que
	\begin{equation}
	(I_x)^{\mu}{}_{\nu}=\begin{pmatrix}
								0 & -1 & 0 & 0\\
								-1 & 0 & 0 & 0 \\
								0 & 0 & 0 & 0 \\
								0 & 0 & 0 & 0
								\end{pmatrix}
	 \; (I_y)^{\mu}{}_{\nu}=\begin{pmatrix}
								0 & 0 & -1 & 0\\
								0 & 0 & 0 & 0 \\
								-1 & 0 & 0 & 0 \\
								0 & 0 & 0 & 0
								\end{pmatrix}
	 \;	
		(I_z)^{\mu}{}_{\nu}=\begin{pmatrix}
								0 & 0 & 0 & -1\\
								0 & 0 & 0 & 0 \\
								0 & 0 & 0 & 0 \\
								-1 & 0 & 0 & 0
								\end{pmatrix}
	\end{equation}
	Adicionalmente, las matrices $I_{x,y,z}$ satisfacen:
	\begin{equation}
	I^2=\begin{pmatrix} 1&0&0&0\\ 0&1&0&0\\ 0&0&0&0\\ 0&0&0&0 \end{pmatrix} \; \text{ y } \; I^3=I
	\end{equation}
	Utilizando la propiedad anterior se itera (\ref{BoostInfX}) con $\Delta \omega=\frac{\omega}{N}$ para recuperar el \textit{boost} de velocidad $v$ en la dirección $x$:
%\left(\mathbb{1}+\frac{\omega}{N}I_x \right)\stackrel{N\rightarrow \infty}{\cdots} \left( \mathbb{1}+\frac{\omega}{N}I_x \right)	
	\begin{align*}
	\Lambda &= \lim_{N\rightarrow \infty}\left( \mathbb{1}+\frac{\omega}{N}I_x\right)^N\\
	&= \exp{\left(\omega I_x\right)} =\cosh{\left( \omega I_x\right)}+\sinh{\left( \omega I_x\right)}\\
	&=\mathbb{1}- I_x^2+I_x^2 \cosh{\omega}+I_x \sinh{\omega}\\	
	&=\begin{pmatrix}
    \cosh{\omega} & -\sinh{\omega} & 0 & 0 \\
    -\sinh{\omega} & \cosh{\omega} & 0 & 0 \\
    0 & 0 & 1 & 0 \\
    0 & 0 & 0 & 1
	\end{pmatrix}
	\end{align*}
	correspondiente a un \textit{boost} con $v=\tanh{\omega}$ y $\cosh{\omega}=\frac{1}{\sqrt{1-v^2}}$. La transformación espinorial (\ref{eq:SFinita}) correspondiente al \textit{boost} es:
	\begin{equation}\label{BoostEspinorial}
\begin{aligned}
\psi^{\prime}(x^{\prime})=S(\Lambda)\psi(x)&=\,\exp{\left( -\frac{i}{4}\omega\left( \sigma_{ 01}{I_{x}}^{01} +\sigma_{10}{I_{x}}^{10}\right) \right)} \psi(x)\\
&=\,\exp{\left( -\frac{i}{2}\omega\sigma_{01} \right)} \psi(\Lambda^{-1} x)\\
&=\,\exp{\left( -\frac{i}{2}\omega \left[ \gamma_0, \gamma_1 \right] \right)} \psi(\Lambda^{-1} x^\prime)
\end{aligned}
\end{equation}
	%Con mayor generalidad, para un \textit{boost} de magnitud $v$ en la dirección $\bm{\hat{v}}$:
\end{example}
\begin{example}
Rotación de ángulo $\varphi$ alrededor de $x_3$. La transformación infinitesimal es
	\begin{equation}\label{RotInf3}
		a^{\mu}{}_{\nu}=	\begin{pmatrix}
							1 & 0 & 0 & 0 \\
							0 & 1 & -\Delta \varphi & 0 \\
							0 & \Delta \varphi & 1 & 0 \\
							0 & 0 & 0 & 1	
							\end{pmatrix}
		= \mathbb{1}+\Delta \varphi \begin{pmatrix}
								0 & 0 & 0 & 0\\
								0 & 0 & -1 & 0 \\
								0 & 1 & 0 & 0 \\
								0 & 0 & 0 & 0
								\end{pmatrix}
	\end{equation}
	y por lo tanto
	\begin{equation}
	(I_3)^{\mu}{}_{\nu}=\begin{pmatrix}
								0 & 0 & 0 & 0\\
								0 & 0 & -1 & 0 \\
								0 & 1 & 0 & 0 \\
								0 & 0 & 0 & 0
								\end{pmatrix}
	\end{equation}
	Las matrices $I_1, I_2$ que generan rotaciones alrededor de los ejes $x_1$ y $x_2$ son análogas; $I_3$ además satisface
	\begin{equation}
	I_3^2=\begin{pmatrix} 0&0&0&0\\ 0&-1&0&0\\ 0&0&-1&0\\ 0&0&0&0 \end{pmatrix}, \; I_3^3=-I_3, \text{ y } \; I_3^4=-I_3^2
	\end{equation}\footnote{Obsérvese que $I_3$ y $I_3^4$ se comportan como las versiones matriciales de $i$ y $1$, respectivamente.}
	 Aplicando sucesivamente (\ref{RotInf3}) se recuperan las rotaciones usuales:
	\begin{align*}
	\Lambda &= \lim_{N\rightarrow \infty}\left( \mathbb{1}+\frac{\omega}{N}I_3\right)^N\\
	&= \exp{\left(\omega I_3\right)}=I_3^4 \cos{\omega}+I_3 \sin{\omega}\\
	&=\begin{pmatrix}
	    1 & 0 & 0 & 0 \\
    0 & \cos{\omega} & -\sin{\omega} & 0 \\
    0& \sin{\omega} & \cos{\omega} & 0 \\
    0 & 0 & 0 & 1
	\end{pmatrix}
	\end{align*}
	En el presente caso $\omega=\varphi$. La transformación espinorial correspondiente a la rotación es:
	\begin{equation}\label{RotEsp}
\begin{aligned}
\psi^{\prime}(x^{\prime})=S(\Lambda)\psi(x)&=\,\exp{\left( -\frac{i}{4}\varphi\left( \sigma_{ 12}{I_{3}}^{12} +\sigma_{21}{I_{3}}^{21}\right) \right)} \psi(x)\\
&=\,\exp{\left( \frac{i}{2}\varphi\sigma_{12} \right)} \psi(\Lambda^{-1} x)=\,\exp{\left( \frac{i}{2}\varphi\sigma^{12} \right)} \psi(\Lambda^{-1} x)\\
&=\,\exp{\left( \frac{i}{2}\varphi \Sigma_3 \right)} \psi(\Lambda^{-1} x^\prime)
\end{aligned}
\end{equation}
con
\begin{equation}\label{SigmaRotacion}
\sigma^{ij}=\left[ \gamma^i,\gamma^j\right]\equiv\Sigma_k\equiv	\begin{pmatrix}
			\sigma_k & 0 \\
			0 & \sigma_k
			\end{pmatrix} \quad (i,j,k \text{ cícilicos})
\end{equation}
La generalización de (\ref{RotEsp}) para una rotación general de ángulo $\left| \bm{\varphi}\right|$ en la dirección $\bm{\hat{\varphi}}$ es directa:
\begin{equation}\label{eq:RotacionEspinorial}
\psi^{\prime}(x^{\prime})=\,\exp{\left( \frac{i}{2}\bm{\varphi}\cdot \bm{\Sigma} \right)} \psi(\Lambda^{-1} x^\prime)
\end{equation}
	\end{example}

La presencia del factor $\sfrac{1}{2}$ en (\ref{eq:RotacionEspinorial}) es característica de las transformaciones espinoriales y tiene por consecuencia el que se requiera de una rotación de $4\pi$, y no $2\pi$, para retornar $\psi$ a su estado inicial, por lo que se hace evidente la necesidad de potencias pares ---u órdenes pares superiores--- para la construcción de observables físicos a partir de $\psi$. 

Las transformaciones (\ref{BoostEspinorial}) y (\ref{eq:RotacionEspinorial}) satisfacen: \begin{equation}\label{eq:SHerm}
S^{-1}=\gamma_0 S^{\dagger} \gamma_0,
\end{equation}
consecuencia de (\ref{eq:CondCoefSigma}) al expandir $S$ como serie de potencias.
Es finalmente posible establecer la covariancia de la 4-corriente $j^{\mu}=\overline{\psi}\gamma^{\mu}\psi$ ---ver subsección \ref{subsec:ContinuidadDirac}---:

\begin{equation}
\begin{aligned}
{j^{\mu}}^{\prime}(x^\prime)&={\psi^{\prime}}^{\dagger}(x^\prime)\gamma^0 \gamma^{\mu}\psi^\prime (x^\prime)\\
&=\psi^{\dagger}(x)S^{\dagger} \gamma^0 \gamma^{\mu} S \psi(x)\\
&=\psi^{\dagger}(x)\gamma^0 \left( S^{-1} \gamma^{\mu} S \right) \psi(x)\\
&=\psi^{\dagger}(x)\gamma^0 \Lambda^{\mu}{}_{\nu}\gamma^{\nu} \psi(x)\\
&=\Lambda^{\mu}{}_{\nu} j^{\nu} (x)
\end{aligned}
\end{equation}
%\label{sec:IdentidadDeGordon}
donde se ha utilizado la condición fundamental (\ref{eq:CondS}) al pasar del tercer al cuarto renglón. Se concluye que la corriente $j^\mu$ es un 4-vector y la ecuación de continuidad $\partial_\mu j^\mu =0$ es invariante ante transformaciones de Lorentz.

%\subsection{How to Leave Comments}
%
%Comments can be added to the margins of the document using the \todo{Here's a comment in the margin!} todo command, as shown in the example on the right. You can also add inline comments:
%
%\todo[inline, color=green!40]{This is an inline comment.}

%\cleardoublepage