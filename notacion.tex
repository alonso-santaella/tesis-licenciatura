\chapter*{Notación}
\label{chap:NOT}
\addcontentsline{toc}{chapter}{Notación}

\small{
\noindent A continuación se presenta un listado de las convenciones de notación, signo, etc., utilizadas en el presente texto, con el objetivo de eliminar ambigüedades asociadas a la multiplicidad de éstas en actual utilización en la literatura del tema. 
\begin{itemize}

	\item Se utilizan unidades tales que $\hbar=c=1$, siendo las primeras secciones del capítulo 1 la excepción a esta regla. (\textit{Unidades naturales})

	\item La métrica de Minkowski es:
	\begin{equation*} 
		g^{\mu\nu}=	\begin{pmatrix}
					1 & 0 & 0 & 0 \\
					0 & -1 & 0 & 0 \\
					0 & 0 & -1 & 0 \\
					0 & 0 & 0 & -1 \\
					\end{pmatrix},
	\end{equation*}
	
	\noindent en coincidencia con \citep{Itzykson}, \citep{Bjorken} y \citep{GreinerRQM}, por ejemplo, pero en discrepancia con \citep{Lurie}, \citep{RobinsonSymmetry} y \citep{WeinbergQTF}. Además, $g^{\mu \nu}=g_{\mu \nu}=(g^{-1})^{\mu \nu}$.
	
	\item Índices en letras latinas (e.g. $i$, $j$, $k$, etc.) toman valores en $\{1,2,3\}$ mientras que índices en letras griegas (e.g. $\alpha$, $\beta$, $\mu$, $\nu$, etc.) toman valores en $\{0,1,2,3\}$. Excepciones a lo anterior se harán explícitas.

	\item Se asume la suma sobre parejas de índices repetidos en una misma expresión (e.g. $u_\alpha v^{\alpha} \equiv \sum_\alpha u_\alpha v^{\alpha} $). (\textit{Convención de suma de Einstein})
	
	\item Los vectores contravariantes se escriben con índices arriba: $x^{\mu}=(x^0,x^1,x^2,x^3)$; los vectores covariantes con índices abajo: $x_{\mu}=(x_0,x_1,x_2,x_3)=(x^0,-x^1,-x^2,-x^3)=g_{\mu \nu}x^{\nu}$. La 4-posición y el 4-momento son: $x^{\mu}=(x^0,x^1,x^2,x^3)=(t,x,y,z)$ y $p^{\mu}=(p^0,p^1,p^2,p^3)=(E,p_x,p_y,p_z)=(E,\mathbf{p})$, respectivamente. 
	
	\item Se define $\partial_\mu=\frac{\partial}{\partial x^{\mu}}$, con lo cual $\partial^2=\partial_\mu \partial^{\mu}= \frac{\partial^2}{\partial t^2}-\nabla^2$ es el operador D\textquotesingle Alambertiano; el \textit{operador} de 4-momento es: $\hat{p}^\mu= i\partial^\mu=i\frac{\partial}{\partial x_\mu}=(i\frac{\partial}{\partial t},-i\mathbf{\nabla})=(\hat{E},\mathbf{\hat{p}})$.
	
\end{itemize}
}